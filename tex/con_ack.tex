\chapter{終わりに}\chaplab{conclusion}
本研究では,ソースコードの可視化およびゲーミフィケーションにより,SATD の除去を支援するシステムを提案した.
提案システムを開発現場に適用することにより,開発者のSATD の除去作業に対する意欲が向上してより多くのSATD が除去されることが期待でき,ソフトウェアの保守性の向上につながると考えられる.
ゲーミフィケーションは金銭的なインセンティブを利用することなく作業者を動機づけることができる手法である.
そのため,提案システムはボランティアで開発が行われるOSS プロジェクトに特に適していると考えられる.

\newpage

\chapter*{謝辞}
\addcontentsline{toc}{chapter}{謝辞}
感謝、感謝です。