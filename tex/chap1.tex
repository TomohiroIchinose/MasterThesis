\chapter{はじめに}
技術的負債とは,ソフトウェア開発における場当たり的な対応やその結果を指す比喩である.
技術的負債は,短期的には開発スピードを速める利点がある一方,長期的に見るとソースコードの保守性を低下させ,開発スピードを遅くするという問題がある[1].
開発者が意識的にソースコード上に残す技術的負債はself-admitted tech-nical debt(以降,SATD)と呼ばれる.
SATD はTODO,FIXMEといったキーワードを含むコメントでその存在が確認できる.
Wehaibi らのSATD とソフトウェアの品質に関する研究では,SATDが含まれているファイルに対する変更(SATD change)と含まれていないファイルに対する変更(non-SATD change)の2 種類を比較している.
複雑度の比較では,non-SATD change よりもSATDchange の方が変更された行数やファイル構造数が多く複雑であるという結果が得られており,SATD はソフトウェアの保守性を低下させているといえる[2].
ソフトウェアの保守性を向上させるには,ソースコード上に残り続けているSATD を除去することが必要だと考えられる.
しかし,ソフトウェアの開発期間中は継続的にSATD がソースコード上に混入し,26.25\%から63.45\%は取り除かれるが,全体としてSATD はソフトウェアに残り続ける傾向があると報告されている[3].
ソースコード上に存在するSATD は,複雑すぎる関数や応急処置的な実装といった,ソフトウェアの動作そのものには影響しないものが大半を占めている[4].
よって,多くのSATD は除去の優先度が低いためにソースコード上に残り続けていると考えられる.
そのため,開発者に優先度の低いSATD を除去を促す仕組みがあれば,ソフトウェアの保守性の向上に繋がると考えられる.
SATD を除去するためにはSATD が存在するソースファイルを知る必要がある.
しかし,SATD ソースコード上のコメントを見なければ存在が分からないため,どのファイルにSATD が存在するかを把握しづらいという問題がある.
効率良くSATD を除去するには,SATD の存在するファイルが分かりやすく可視化されることが望ましい.
本研究では,ソースコード上に残っているSATD の除去を支援するための,ゲーミフィケーション,およびソースコード可視化を利用したシステムを提案する.
提案システムでは,ランキング提示による競争を促すゲーミフィケーションを用いることで,積極的なSATD の除去を開発者に促し,ソフトウェアの保守性の維持向上を目指す.
また,提案システムはソースコードのファイル構造を街のように可視化し,SATD が存在するファイルを目立たせることで,除去するべきSATD を分かりやすくユーザに提示する.