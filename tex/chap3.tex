\chapter{提案システム}\chaplab{result}
提案システムはGitで管理されているリポジトリを対象とする.
Python プログラムでリポジトリを解析し,街の情報を生成する.
各開発者のSATD の除去数情報はプロジェクト管理者用ツール上で編集し,街の情報に付与する.
街はゲームエンジンの一つであるUnity上で可視化する.

\section{システム構成}
サーバ側含めたシステム全体の構成説明.

\section{リポジトリ解析}
リポジトリ解析はPython ライブラリの一つであるGitPythonを利用した,約180 行のPython プログラムで実施する.
解析プログラムはGit リポジトリに存在するソースファイルを探索し,街の情報としてソースファイル名や存在するディレクトリ名,コード行数,コメント行数,SATD が存在する行番号を取得する.
対応しているプログラミング言語はJava,Ruby,Python の3種類である.STAD はPotdarらの定義する62個のキーワードから判断する[3].
解析結果はJson ファイルとして出力される.

\section{Webシステム}
Webシステムの説明.

\subsection{Rocat}
(ACITの論文引いて)Rocatを説明.

\subsection{ソースコードとSATDの情報提示}
元のRocatからの改良点他,HEIZOでの可視化部分の説明.

\subsection{ランキング}
ランキング部分の説明.