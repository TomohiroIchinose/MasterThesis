\chapter{関連研究}\chaplab{definition}
Potdar らはソフトウェアに存在するSATD の個数を明らかにするため,SATD を示す62 個のコメントパターンを定義し,4つのOSS プロジェクトに存在するSATD の個数を調査している.
その調査では,SATD がソースファイル上に2.4\%から31.0\%含まれているという結果が得られている[3].
da S. Maldonado らはSATD の種類ごとの個数を明らかにするため,SATD を関連する5 種類の問題点(設計・欠陥・文書化・要求物・テスト)ごとに分類している.
5つのOSS プロジェクトのソースコードの調査では,最も割合が高いのは設計に関するSATD であり,42\%から84\%を占めているという結果が得られている[4].
設計に関するSATD は複雑すぎる関数や応急処置的な実装といったものであり,ソフトウェアの動作に影響する欠陥や要求物に関するSATD に比べると除去の優先度が低いと考えられる.
SATD の除去に対し,何らかのインセンティブを設定することにより,開発者に優先度の低いSATD を除去させられる可能性がある.
本研究ではゲーミフィケーションを利用することにより,ソースコード上に残っている,除去優先度の低いSATD の除去を促すシステムを提案する.


ソフトウェアの可視化は,プログラム読解支援や問題検出の結果提示など,様々な目的に対して数多く研究されている[5][6].
Wettel らの提案するCodeCity は,ソースコードのクラスとパッケージの構造を街のように3D で可視化するシステムである.
街の構造は直感的で親しみやすく,ソフトウェアの複雑な構造を単純化しすぎることなく表現するのに適している[7].
Balogh らはコンピュータゲームであるMineCraftを用いてソースコードを街のように可視化するCodeMetropolis を利用し,ソフトウェアテストに関連するメトリクスを可視化するシステムを提案している[8].
Balogh らのシステムでは関連のあるテストケースとソースコードを並べて配置して可視化することで,開発者の理解を支援している.
本研究では街の構造を利用した可視化を用い,SATDのあるソースコードに目印を付けることによって,開発者による,SATD があるソースコードの特定作業を支援する.


近年では,シリアスゲームやGames with a purpose といったゲームを問題解決に利用する手法への関心が高まっている[9][10].
その中でもゲーミフィケーションはゲームそのものではなく,ゲームの要素を社会活動や,活動を支援するシステムに取り入れる事で意欲を維持向上させる手法である[11].
ゲーミフィケーションによって作業を支援するシステムはこれまでに複数提案されている[12][13][14].
また,ソフトウェア開発への適用も注目されている[15][16].既存研究で提案されたシステムでは,ゲームの要素として競争がよく用いられている.
競争はカイヨワの提案する楽しさを生む4 つの要因[17] や,Steven が定義した人を動機づける16 個の欲求[18] などにおいても見られるため,ゲーミフィケーションに用いる要素として適しているといえる.
本研究では,システム内でSATD の除去数をランキング形式で提示し,開発者同士で競わせて動機づけることでSATD の積極的な除去を支援する.